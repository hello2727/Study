<MySQL 입장>
우선, cmd창을 키고 밑의 명령어를 칩니다.
패스워드 필요 : mysql -uroot -p 
패스워드 필요 x : mysql -uroot

<비밀번호 재설정>
SET PASSWORD = PASSWORD('바꿀 루트 비밀번호')

<데이터 베이스 생성>
CREATE DATABASE 만들 데이터베이스 이름;

<데이터 베이스 제거>
DROP DATABASE 만들어진 데이터베이스 이름;

<만들어진 데이터베이스 목록 보기>
SHOW DATABASES;

<만들어진 데이터베이스 목록 중 특정 데이터베이스 사용>
USE 만들어진 데이터베이스 이름;

<테이블 생성>
CREATE TABLE 만들 테이블 이름(
  원하는컬럼이름 데이터형식(노출 길이) 옵션,
             ''                        ,
             ''                        , 
  PRIMARY KEY(id)      ->괄호 값 중복 허용 안함     
); 
*[옵션 필기] 
NOT NULL:빈칸 허용 안됨, AUTO_INCREMENT:자동 증가, NULL:빈칸 가능

<만들어진 테이블 목록 보기>
SHOW TABLES;

<만들어진 테이블 구조 보기>
DESC 만들어진테이블;

<테이블 안에 레코드(=ROW)값 삽입>
INSERT INTO 만들어진테이블 (파라미터1, ..., 파라미터N) VALUES('문자일경우 문자값', 숫자일 경우 숫자값, 데이터 형식에 알맞은 값, 설정된 파라미터 수대로);

<만들어진 테이블로부터 컬럼(=COLOMN) 값 가져오기>
SELECT * FROM 만들어진테이블; //모든 column값 가져오기

<만들어진 테이블로부터 특정 컬럼(=COLOMN) 값 가져오기>
SELECT 특정컬럼(','로 다중 선택 가능) FROM 만들어진테이블; 
*[중요]
SELECT "egoing"; //SELECT뒤엔 항상 컬럼이 나와야 됨. 'FROM'은 생략 가능

<만들어진 테이블로부터 특정 컬럼(=COLOMN) 값 가져오는데, 그 중 원하는 컬럼의 값과 일치하는 레코드(=row)만 출력>
SELECT 특정컬럼(','로 다중 선택 가능) FROM 만들어진테이블 WHERE 특정컬럼='특정컬럼로우값'; 

<만들어진 테이블로부터 특정 컬럼(=COLOMN) 값 가져오는데, 그 중 원하는 컬럼의 값과 일치하는 레코드(=row)만 출력&&내림차순 정렬>
SELECT 특정컬럼(','로 다중 선택 가능) FROM 만들어진테이블 WHERE 특정컬럼='특정컬럼로우값' ORDER BY 첫번째컬럼(보통 id값) DESC; 

<만들어진 테이블로부터 특정 컬럼(=COLOMN) 값 가져오는데, 그 중 원하는 컬럼의 값과 일치하는 레코드(=row)만 출력&&내림차순 정렬&&출력수제한>
SELECT 특정컬럼(','로 다중 선택 가능) FROM 만들어진테이블 WHERE 특정컬럼='특정컬럼로우값' ORDER BY 첫번째컬럼(보통 id값) DESC LIMIT 원하는슷자;

<만들어진 테이블 업데이트>
UPDATE 만들어진테이블 SET 업데이트원하는특정컬럼='바꾸려는내용'(여러개를','이용하여 선택가능) WHERE 컬럼1=업데이트원하는레코드행숫자;

<만들어진 테이블중 선택한 행 삭제>
DELETE FROM 만들어진테이블 WHERE 컬럼1 = 삭제원하는레코드행숫자;

----------------------------------------------------관계형데이터베이스---------------------------------------------------------
